% 1. Introducción a los códigos lineales
% Por ahora es el capítulo 1 de Huffman y Pless

\chapter{Introducción a los códigos lineales}

Todo el desarrollo de este capítulo está basado en \cite{Huffman_Pless_2010} .
\section{Códigos lineales}

Sea $\mathbb{F}_q$ el cuerpo finito de $q$ elementos, denotamos $\mathbb{F}_q^n$ al espacio vectorial de las n-tuplas sobre el cuerpo finito $\mathbb{F}_q$. A los vectores $(a_1,a_2,\cdots,a_n)$ de $\mathbb{F}_q$ generalmente los escribiremos como $a_1a_2\cdots a_n$.



\begin{definition}
Un $(n,M)$ \emph{código} $\mathcal{C}$ sobre $\mathbb{F}_q$ es un subconjunto de $\mathbb{F}_q^n$ de tamaño $M$. Llamaremos \emph{palabras código} a los elementos de $\mathcal{C}$.
\end{definition}

\begin{exampleth}
\begin{itemize}
	\item En el cuerpo $\mathbb{F}_2$, a los códigos se les conoce como \emph{códigos binarios} y un ejemplo sería $\mathcal{C} = \{00,01,10,11\}$.

	\item En el cuerpo $\mathbb{F}_3$, a los códigos se les conoce como \emph{códigos ternarios} y un ejemplo sería $\mathcal{C} = \{01, 12, 02, 10, 20, 21, 22\}$.
\end{itemize}
\end{exampleth}

Si $\mathcal{C}$ es un espacio k-dimensional de $\mathbb{F}_q^n$, entonces decimos que $\mathcal{C}$ es un $\left[n,k\right]$ \emph{código linear} sobre $\mathbb{F}_q$ y tiene $q^k$ palabras código. Las dos formas más comunes de representar un código lineal son con la \emph{matriz generadora} o la \emph{matriz de paridad}.

\begin{definition}

Una \emph{matriz generadora} de un $\left[n,k\right]$ \emph{código linear} $\mathcal{C}$ es cualquier matriz $k \times n$ cuyas columnas forman una base de $\mathcal{C}$.
\end{definition}

Para cada conjunto de $k$ columnas independientes de una matriz generadora $G$, se dice que el conjunto de coordenadas forman un \emph{conjunto de información} de $\mathcal{C}$. Las $r = n - k$ coordenadas restantes forman el \emph{conjunto de redundancia} y el número $r$ es la \emph{redundancia} de $\mathcal{C}$.

En general no hay una única matriz generadora pero si las primeras $k$ coordenadas forman un conjunto de información, entonces el código tiene una única matriz generado de la forma $\left[I_k | A\right]$, donde $I_k$ es la matriz identidad $k \times k$. Esta matriz se dice que está en \emph{forma estándar}.

Como un código linear es un subespacio de un espacio vectorial, es el núcleo de alguna transformación lineal.

\begin{definition}
Una \emph{matriz de paridad} $H$ de dimensión $(n-k) \times k$ de un $\left[n,k\right]$ \emph{código linear} $\mathcal{C}$ es una matriz que verifica :
\[
C = \{ x \in  \mathbb{F}_q^n | Hx^T = 0  \}
\]
\end{definition}

Como ocurría con la matriz generadora, la matriz de paridad no es única. Con el siguiente resultado podremos obtener una de ellas cuando $\mathcal{C}$ tiene una matriz generadora en forma estándar.

\begin{theorem}[Matriz de paridad a partir de la generadora]

Si $G = \left[I_k | A\right]$ es una matriz generadora del $\left[n,k\right]$ código $\mathcal{C}$ en su forma estándar, entonces $H = \left[-A^T |I_{n-k}\right]$ es la matriz de paridad de $\mathcal{C}$.
\end{theorem}

\begin{proof}
Sabemos que $HG^T = -A^T + A^T = 0$, luego $\mathcal{C}$ está contenido en el núcleo de la transformación lineal $x \mapsto Hx^T$. Como $H$ tiene rango $n-k$, el núcleo de esta transformación es de dimensión $k$ que coincide con la dimensión de $\mathcal{C}$.
\end{proof}

\begin{exampleth}
 \label{ex:matriz_generadora}
Sea la matriz $G = \left[I_4 | A\right]$, donde

\[
G = \left( \begin{array}{cccc|ccc}
	1 & 0 & 0 & 0 & 0 & 1 & 1 \\
	0 & 1 & 0 & 0 & 1 & 0 & 1 \\
	0 & 0 & 1 & 0 & 1 & 1 & 0 \\
	0 & 0 & 0 & 1 & 1 & 1 & 1 
			\end{array} 
	\right)
\]
es la matriz generadora en forma estándar del $\left[7,4\right]$ código binario que denotaremos por $\mathcal{H}_3$.
Por el teorema, la matriz de paridad de $\mathcal{H}_3$ es

\[
H =  \left[A^T | I_3\right] = \left( \begin{array}{cccc|ccc}
	0 & 1 & 1 & 1 & 1 & 0 & 0 \\
	1 & 0 & 1 & 1 & 0 & 1 & 0 \\
	1 & 1 & 0 & 1 & 0 & 0 & 1 

			\end{array} 
	\right)
\]

Este código se le conoce como el $\left[7,4\right]$ \emph{código de Hamming}.
\end{exampleth}

\section{Códigos duales}
La matriz generadora $G$ de un $\left[n,k\right]$ código $\mathcal{C}$  es simplemente una matriz cuyas filas son independientes y que expanden el código. Las filas de la matriz de paridad $H$ también son independientes, luego $H$ es la matriz generadora del mismo código al que llamaremos \textit{código dual u ortogonal} y lo denotaremos como $\mathcal{C}^\perp$. Notamos que $\mathcal{C}^\perp$ es un $\left[n,n-k\right]$ código. Otra forma de verlo es de la siguiente manera: \\


\begin{definition}
$\mathcal{C}$ es un subespacio de un espacio vectorial luego a su ortogonal es a lo que llamamos \textit{espacio dual u ortogonal} de $\mathcal{C}$ y viene dado por

$\mathcal{C}^\perp = \left\{ \textbf{x} \in \mathbb{F}_q^n \; : \; \textbf{x} \cdot \textbf{c} = 0 \quad  \forall \textbf{c} \in \mathcal{C} \right\}$
\end{definition}

Vamos a obtener ahora la matriz generadora y de paridad de $\mathcal{C}^\perp$ a partir de las de $\mathcal{C}$

\begin{proposition}
Si $G$ y $H$ son las matrices generadora y de paridad de $\mathcal{C}$ respectivamente, entonces $H$ y $G$ son las matrices generadora y de paridad de $\mathcal{C}^\perp$.
\end{proposition}

\begin{proof}
Sea $G = \left[I_k | A\right]$ la  matriz generadora y $H = \left[-A^T |I_{n-k}\right]$ la matriz de paridad del $\left[n,k\right]$ código $\mathcal{C}$. 

Sabemos que $HG^T = GH^T = 0$ luego

$\mathcal{C}^\perp = \left\{ \textbf{x} \in \mathbb{F}_q^n \; : \; \textbf{x} \cdot \textbf{c} = 0 \quad  \forall \textbf{c} \in \mathcal{C} \right\} = \left\{ \textbf{x} \in \mathbb{F}_q^n \; : \; \textbf{x} \cdot G^T = 0 \quad  \forall \textbf{c} \in \mathcal{C} \right\} = $ \\
 $\qquad \qquad = \left\{ \textbf{x} \in \mathbb{F}_q^n \; : \; G \cdot \textbf{x}^T = 0 \quad  \forall \textbf{c} \in \mathcal{C} \right\}$
 
Luego $\mathcal{C}^\perp$ está contenido en el núcleo de la transformación lineal $x \mapsto Gx^T$. Como $G$ tiene rango $k$, el núcleo de esta transformación es de dimensión $n-k$ que coincide con la dimensión de $\mathcal{C}^\perp$. Por tanto, $G$ es la matriz de paridad de $\mathcal{C}^\perp$.

Por último, como $HG^T = 0$ entonces $H$ es la matriz generadora de $\mathcal{C}^\perp$.
\end{proof}

Tras este resultado se ve claramente que $\mathcal{C}^\perp$ es un $\left[n,n-k\right]$ código.

\begin{definition}
Diremos que un código $\mathcal{C}$ es auto-ortogonal si $\mathcal{C} \subseteq \mathcal{C}^\perp$ y diremos que es autodual si $\mathcal{C} = \mathcal{C}^\perp$

\end{definition}

\begin{exampleth}
Tenemos una matriz generadora del código de Hamming $\left[7,4\right]$ dada en el ejemplo \ref{ex:matriz_generadora}. Ahora definimos $\mathcal{H}'_3$ como el $\left[8,4\right]$ código en donde hemos añadido una columna a la paridad de $G$. Sea

\[
G' = \left( \begin{array}{cccc|cccc}
	1 & 0 & 0 & 0 & 0 & 1 & 1 & 1\\
	0 & 1 & 0 & 0 & 1 & 0 & 1 & 1\\
	0 & 0 & 1 & 0 & 1 & 1 & 0 & 1 \\
	0 & 0 & 0 & 1 & 1 & 1 & 1 & 0
			\end{array} 
	\right)
\]

donde $G'$ es la matriz generadora de $\mathcal{H}'_3$. Veamos que es autodual:

Sabemos que $G' = \left[I_4 | A'\right]$  y en este caso $A'$ es la siguiente matriz:

\[
A' = \left( \begin{array}{cccc}
	 0 & 1 & 1 & 1\\
	 1 & 0 & 1 & 1\\
	 1 & 1 & 0 & 1 \\
	 1 & 1 & 1 & 0
			\end{array} 
	\right)	
\]

y $(A')^T$ es la misma matriz. Luego como $A'(A')^T = I_4 $ entonces $\mathcal{H}'_3$ es autodual.

\end{exampleth}

\section{Pesos y distancias}
\begin{definition}
La \textit{distancia de Hamming} $d(x,y)$ entre dos vectores $x,y \in \mathbb{F}_q^n$ es el número de coordenadas en las que x e y difieren. 
\end{definition}

\begin{exampleth}
	Sea $\mathbf{x}=20110$ y $\mathbf{y}=10121$ entonces $d(x,y)=3$.
\end{exampleth}

\begin{theorem}
	La función distancia $d(x,y)$ satisface las siguientes cuatro propiedades:
	\begin{enumerate}
	\item No negatividad: $d(x,y) \geq 0 \quad \forall x,y \in \mathbb{F}_q^n$.
	\item $d(x,y)=0 \Leftrightarrow x = y$.
	\item Simetría: $d(x,y)=d(y,x) \quad \forall x,y \in \mathbb{F}_q^n$.
	\item Desigualdad triangular: $d(x,z)\leq d(x,y) + d(y,z) \quad \forall x,y,z \in \mathbb{F}_q^n$
	\end{enumerate}
	
\end{theorem}

\begin{proof}
Las tres primeras propiedades son evidentes por la definición de la distancia, comprobemos la última propiedad.

Distinguimos dos casos, si $ x = z $ tenemos que $d(x,z) = 0$ y por tanto se verifica la desigualdad. Si $x \neq z$ entonces, no puede ocurrir que $x = y = z $, por tanto $d(x,y) \neq 0$ o $d(y,z) \neq 0$ y por la no negatividad se tendría la desigualdad, en el caso de que $ x = y$ o $y = z$ tendríamos la igualdad.
\end{proof}

Llamaremos \textit{distancia mínima} de un código $\mathcal{C}$ a la menor distancia no-nula entre dos palabras cualquiera del código. Además, esta distancia es un invariante y es importante a la hora de determinar la capacidad de corrección de errores del código $\mathcal{C}$ .

\begin{exampleth}
Sea $\mathcal{C} = \left\{ 201310, 311210, 202210, 312100 \right\} $ un código. Sus distancias son:
\[
d(201310, 311210) = 3 , \quad d(201310,202210) = 2, \quad d(201310,312100) = 5,
\]
\[
 d(311210, 202210) = 3, \quad d(311210,312100) = 3, \quad d(202210,312100) = 4 
\]
Luego, la distancia mínima es $d(\mathcal{C}) = 2$.
\end{exampleth}

\begin{definition}
El \textit{peso de Hamming} o $\operatorname{wt}(x)$ de un vector $x \in \mathbb{F}_q^n$ es el número de coordenadas no-nulas en $x$. Llamaremos \textit{peso de $\mathcal{C}$} a 
$\operatorname{wt}(\mathcal{C}) = \min(\operatorname{wt}(x))$ con $x \neq 0 $.
\end{definition}

\begin{exampleth}
	Sea $\mathbf{x} = 202210$ un vector en $\mathbb{F}_3^6$ entonces $\operatorname{wt}(x) = 4$.
\end{exampleth}

\begin{theorem}
Si $x,y \in \mathbb{F}_q^n$, entonces $d(x,y) = \operatorname{wt}(x-y)$. Si $\mathcal{C}$ es un código linear, la mínima distancia $d$ es igual al mínimo peso de $\mathcal{C}$.
\end{theorem}

\begin{proof}
Como $\mathcal{C}$ es lineal, tenemos que $ 0 \in \mathcal{C}$ y además $\operatorname{wt}(x) = d(x,0) \quad \forall x \in \mathcal{C}$, luego $d(\mathcal{C}) \leq \operatorname{wt}(\mathcal{C})$.

Por otro lado, sea $x,y \in \mathcal{C}$ entonces $x-y \in \mathcal{C} \quad \forall x,y \in \mathcal{C}$ y sabemos que $d(x,y) = \operatorname{wt}(x-y) \geq \operatorname{wt}(\mathcal{C})$ para cualesquiera $x,y \in \mathcal{C}$. Se tiene que $d(\mathcal{C}) \geq \operatorname{wt}(\mathcal{C})$.

Hemos conseguido así la igualdad, $d(\mathcal{C}) = \operatorname{wt}(\mathcal{C})$.
\end{proof}

Como resultado de este teorema, para códigos lineales, la \textit{mínima distancia}  también se denomina el \textit{peso mínimo} de un código. Si se conoce el peso mínimo de un código, entonces nos referiremos a él como el $\left[n,k,d\right]$ código.


\section{Códigos cíclicos}

Vamos a estudiar los códigos cíclicos de longitud n, por ello, denotaremos las coordenadas de sus posiciones como $0,\cdots,n-1$ que son los enteros módulo n.

\begin{definition}
 Un código lineal $\mathcal{C}$ de longitud n sobre $\mathbb{F}_q$ es \textit{cíclico} si para cada vector $c = c_0,\cdots,c_{n-2},c_{n-1}$ en  $\mathcal{C}$, el vector $c_{n-1},c_0,\cdots,c_{n-2}$ obtenido de \textbf{c} por la permutación de las coordenadas $ i \rightarrow i+1 (mod n)$, está también en $\mathcal{C}$.
 \end{definition}
 
 Así, un código cíclico contiene las n permutaciones de cada palabra código. Por tanto, es conveniente pensar que las coordenadas cuando alcanzan $n-1$, vuelven a la coordenada 0.

Cuando hablemos de códigos cíclicos sobre $\mathbb{F}_q$, normalmente las palabras códigos las representaremos en su forma polinómica, ya que hay una correspondencia biyectiva entre los vectores $c = c_0,c_1,\cdots,c_{n-1}$ en $\mathbb{F}_q^n$ y los polinomios $c(x)=c_0+c_1x+\cdots+c_{n-1}x^{n-1}$ en $\mathbb{F}_q[x]$ con grado como mucho $n-1$. Notemos que si \textbf{c} es el polinomio dado, entonces $xc(x)=c_{n-1}x^{n}+c_0x+c_1x^2+\cdots+c_{n-2}x^{n-1}$ representa una permutación de \textbf{c} si $x^n$ es igual a 1. Más formalmente, el hecho de que el código cíclico $\mathcal{C}$ sea invariante por permutaciones, implica que $c(x)$ está en $\mathcal{C}$, luego $xc(x)$ también lo está multiplicando módulo $x^n-1$. 

Esto sugiere que para un mejor estudio de los códigos cíclicos, desarrollemos el anillo cociente \[ \mathcal{R}_n =\mathbb{F}_q[x] / (x^n-1)\]  bajo la correspondencia vectores-polinomios dada anteriormente, los códigos cíclicos son ideales de $\mathcal{R}_n$ y los ideales de $\mathcal{R}_n$ son códigos cíclicos. Luego, el estudio de códigos cíclicos en $\mathbb{F}_q^n$ es equivalente al estudio de los ideales de $\mathcal{R}_n$ que se basa en factorizar el polinomio $x^n-1$


\subsection{Factorización de $x^n-1$ }
Queremos encontrar los factores irreducibles de $x^n-1$. Encontramos dos posibilidades: que $x^n-1$ tenga factores irreducibles repetidos o no los tenga. En el caso de los códigos cíclicos, se centra más en el segundo caso, por ello, hacemos la asumpción de que $x^n-1$ no tiene factores repetidos si y solo si \textit{q} y \textit{n} son primos relativos.

Como ayuda para factorizar $x^n-1$ sobre $\mathbb{F}_q^n$, es útil encontrar una extensión del cuerpo $\mathbb{F}_{q^t}^n$ sobre $\mathbb{F}_q^n$ que contiene todas las raíces del polinomio. En otras palabras, $\mathbb{F}_{q^t}^n$ debe contener las raíces primitivas \textit{n-ésimas} de la unidad, que ocurre cuando $ n \mid (q^t-1)$.  Definimos el orden, $ord_n(q)$ de \textit{q} módulo \textit{n}, como el entero positivo más pequeño \textit{a} tal que $q^a \equiv 1 \thinspace ( mod \thinspace n)$. Notemos que si $t = ord_n(q)$, entonces $\mathbb{F}_{q^t}^n$ contiene la raíz primitiva \textit{n-ésima} de la unidad $\alpha$ , pero ninguna extesión del cuerpo $\mathbb{F}_q^n$ contiene esa raíz. Como los $\alpha^i$ son distintos para $0 \leq i < n $ y $(\alpha^i)^n = 1$ , entonces $\mathbb{F}_{q^t}^n$ contiene todas las raíces de $x^n-1$. Consecuentemente, llamaremos a $\mathbb{F}_{q^t}^n$ el \textit{cuerpo de descomposición} de $x^n-1$ sobre $\mathbb{F}_q^n$. Así que los factores irreducibles de $x^n-1$ sobre $\mathbb{F}_q^n$ deben de ser productos de los distintos polinomios mínimos de las raíces \textit{n-ésimas} en $\mathbb{F}_{q^t}^n$.  Supongamos que $\gamma$ es un elemento primitivo, es decir, el elemento generador de $\mathbb{F}_{q^t}^n$, entonces $\alpha = \gamma^d$ es una raíz primitiva \textit{n-ésima} de la unidad en donde $d = (q^t-1)/n$. Las raíces de $\mathcal{M}_{\alpha^s}(x)$ son $\{ \gamma^{ds}, \gamma^{dsq},\gamma^{dsq^2} ,\cdots, \gamma^{dsq^{r-1}} \} = \{ \alpha^s,\alpha^{sq},\alpha^{sq^2},\cdots,\alpha^{sq^{r-1}} \}$ donde \textit{r} es el entero positivo más pequeño que cumple que $dsq^r \equiv ds \thinspace ( mod \thinspace q-1 )$ pero esto solo se verifica si y solo si $sq^r \equiv s \thinspace ( mod \thinspace n )$
% n | (q t − 1) by Theorem 3.3.3. Define the order ord n (q) of

\begin{definition}
Sea $\mathbb{F}_q^n$ un cuerpo finito y $\mathbb{F}_{q^t}^n$ un cuerpo de extensión suyo, llamaremos \textit{clase q-ciclotómica de s módulo n} al conjunto :
\[
	\mathcal{C}_s = \{s,sq,\cdots,sq^{r-1} \} \thinspace (mod \thinspace n)
\]
donde \textit{r} es el menor entero positivo tal que $sq^r \equiv s \thinspace (mod \thinspace n ) $.
\end{definition}

Las distintas clases \textit{q-ciclotómicas} modulo \textit{n} forman una partición del conjunto de los enteros $\{0,1,2,\cdots,n-1\}$.

\begin{exampleth}
\label{ex: clase_mod_nueve}
 Vamos a calcular las clases \textit{2-ciclotómicas} para $n=9$ y $q=2$ :

La primera de todas es $\mathcal{C}_0 = \{ 0*2^r \equiv 0 \thinspace (mod \thinspace 9) \} = \{ 0 \} $ y repetimos este proceso. Luego tenemos:

$\mathcal{C}_1 = \{ 1*2^r \equiv 1 \thinspace (mod \thinspace 9) \} = \{ 1, 1*2^1=2, 1*2^2 = 4, 1*2^3 = 8, 1*2^4 = 7, 1*2^5 = 5 \} = \{1,2,4,8,7,5\} $
ya que $1*2^6 = 64 \equiv 1 \thinspace (mod \thinspace 9)$ luego $r=6$

$\mathcal{C}_3 = \{ 3*2^r \equiv 3 \thinspace (mod \thinspace 9) \} = \{ 3,3*2 = 6 \} = \{3,6 \} $ ya que $3*2^2 = 12 \equiv 3 \thinspace (mod \thinspace 9 )$ luego $r=2$

\end{exampleth}

\begin{exampleth} 
\label{ex:clase_mod_trece}
Vamos a calcular las clases \textit{3-ciclotómicas} para $n=13$ y $q=3$ que serán las siguientes :

$\mathcal{C}_0 = \{ 0*3^r \equiv 0 \thinspace (mod \thinspace 13) \} = \{ 0 \} $

$\mathcal{C}_1 = \{ 1*3^r \equiv 1 \thinspace (mod \thinspace 13) \} = \{ 1, 1*3^1=3, 1*3^2 = 9 = \{1,3,9\} $
ya que $1*3^3 = 27 \equiv 1 \thinspace (mod \thinspace 13)$ luego $r=3$

$\mathcal{C}_2 = \{ 2*3^r \equiv 2 \thinspace (mod \thinspace 13) \} = \{ 2, 2*3^1=6, 2*3^2 = 5 \} = \{2,6,5\} $
ya que $2*3^3 = 54 \equiv 2 \thinspace (mod \thinspace 13)$ luego $r=3$ 

$\mathcal{C}_4 = \{ 4*3^r \equiv 4 \thinspace (mod \thinspace 13) \} = \{ 4, 4*3^1=12, 4*3^2 = 10 \} = \{4,12,10\} $
ya que $4*3^3 = 108 \equiv 4 \thinspace (mod \thinspace 13)$ luego $r=3$

$\mathcal{C}_7 = \{ 7*3^r \equiv 7 \thinspace (mod \thinspace 13) \} = \{ 7, 7*3^1=8, 7*3^2 = 11 \} = \{7,8,11\} $
ya que $7*3^3 = 189 \equiv 7 \thinspace (mod \thinspace 13)$ luego $r=3$

Luego, ya tenemos todas las clases \textit{3-ciclotómicas} para $n=13$ y $q=3$.

\end{exampleth}



\begin{theorem}
\label{factorizar_xn}
Sea \textit{n} un entero positivo, primo relativo con \textit{q}. Sea $t = ord_n(q)$ y sea $\alpha$ la raíz primitiva \textit{n-ésima} de la unidad en $\mathbb{F}_{q^t}$.
\begin{enumerate}

	\item Por cada entero \textit{s} con $0 \leq s < n$, el polinomio mínimo de $\alpha^s$ sobre  $\mathbb{F}_q$ es
	\[
	\mathcal{M}_{\alpha^s}(x) = \prod_{i \in \mathcal{C}_s} (x-\alpha^i)
	\] 
	donde $\mathcal{C}_s$ es la clase \textit{q-ciclotómica} de \textit{s} módulo \textit{n}
	\item  Los conjugados de $\alpha^s$ son los elementos $\alpha^i$ con $i \in \mathcal{C}_s$
	\item  
	\[
	x^n-1 =  \prod_{s} \mathcal{M}_{\alpha^s}(x) 
	\]
	es la factorización de $x^n-1$ en factores irreducibles sobre $\mathbb{F}_q$ donde \textit{s} recorre un conjunto de los representantes de la clase \textit{q-ciclotómica} modulo \textit{n}.
\end{enumerate}
\end{theorem}
 

 \begin{exampleth}
Vamos a factorizar $x^9-1$ para ello, cogemos las clases \textit{2-ciclotómicas} calculadas en el ejemplo \ref{ex: clase_mod_nueve} que son $\mathcal{C}_0 = \{ 0 \}$ , $\mathcal{C}_1 = \{ 1,2,4,8,7,5 \}$ y $\mathcal{C}_3 = \{ 3,6 \}$. Luego el $ord_9(2) = 6$ y la \textit{nueve-ésima} raíz primitiva de la unidad reside en el cuerpo de extensión $\mathbb{F}_{64}$ y no en ningún otro más pequeño cuerpo de extensión de $\mathbb{F}_2$.

Podemos afirmar que los factores irreducibles de $x^9-1$ tienen grado 1,2 y 6. Estos polinomios son $\mathcal{M}_{1}(x) = x -1 $ , $\mathcal{M}_{\alpha}(x)$ y $\mathcal{M}_{\alpha^3}(x)$ donde $\alpha$ es la \textit{nueve-ésima} raíz primitiva de la unidad en $\mathbb{F}_{64}$. Como el único polinomio irreducible de grado dos en $\mathbb{F}_2$ es $x^2+x+1$ no queda otra que sea $\mathcal{M}_{\alpha^3}(x)$. Por tanto, así tenemos la factorización que es $x^9-1 = (x-1)(x^2+x+1)(x^6+x^3+1)$ y $\mathcal{M}_{\alpha}(x) = x^6+x^3+1$.
 
 \end{exampleth}
 
\begin{exampleth}
Ahora vamos a factorizar $x^13-1$ para ello, cogemos las clases \textit{3-ciclotómicas} calculadas en el ejemplo \ref{ex:clase_mod_trece} que son $\mathcal{C}_0 = \{ 0 \}$ , $\mathcal{C}_1 = \{ 1,3,9 \}$, $\mathcal{C}_2 = \{ 2,6,5 \}$,  $\mathcal{C}_4 = \{ 4,12,10 \}$ y  $\mathcal{C}_7 = \{ 7,8,11 \}$ . Luego el $ord_13(3) = 3$ y la \textit{trece-ésima} raíz primitiva de la unidad reside en el cuerpo de extensión $\mathbb{F}_{27}$ y no en ningún otro más pequeño cuerpo de extensión de $\mathbb{F}_3$.

Podemos afirmar que los factores irreducibles de $x^13-1$ tienen grado 1,3,3,3 y 3. Estos polinomios son $\mathcal{M}_{1}(x) = x -1 $ , $\mathcal{M}_{\alpha}(x)$ y $\mathcal{M}_{\alpha^2}(x)$,  $\mathcal{M}_{\alpha^4}(x)$ y  $\mathcal{M}_{\alpha^7}(x)$ donde $\alpha$ es la \textit{trece-ésima} raíz primitiva de la unidad en $\mathbb{F}_{27}$. 

Como el único polinomio irreducible de grado dos en $\mathbb{F}_2$ es $x^2+x+1$ no queda otra que sea $\mathcal{M}_{\alpha^3}(x)$. Por tanto, así tenemos la factorización que es $x^9-1 = (x-1)(x^2+x+1)(x^6+x^3+1)$ y $\mathcal{M}_{\alpha}(x) = x^6+x^3+1$.

\end{exampleth}
 
Viendo estos ejemplos podemos sacar que el tamaño de cada clase \textit{q-ciclotómica} es un divisor del $ord_n(q)$.

\begin{theorem}
El tamaño de cada clase \textit{q-ciclotómica} es un divisor del $ord_n(q)$. Además, el tamaño de $\mathcal{C}_1$ es justamente el  $ord_n(q)$.
\end{theorem}

\begin{proof}
Sea $t = ord_n(q)$ y sea $m$ el tamaño de $\mathcal{C}_s$. Entonces $\mathcal{M}_{\alpha^s}(x)$ es un polinomio de grado m donde $\alpha$ es la \textit{n-ésima} raíz primitiva de la unidad. Así que, $ m \mid t$. Por definición de orden y clase \textit{q-ciclotómica} sale que el tamaño de $\mathcal{C}_1 = ord_n(q) $.
\end{proof}


\subsection{Teoría básica de los códigos cíclicos}

Anteriormente, denotamos que los códigos cíclicos sobre $\mathbb{F}_q$ son precisamente los ideales de 
\[
\mathcal{R}_n = F_q[x]/(x^n-1)
\] 
Además cada ideal de $\mathbb{F}_q[x]$ es un ideal principal, luego los ideales de $\mathcal{R}_n$ son también principales y por eso, los códigos cíclicos son ideales principales de $\mathcal{R}_n$.

Los elementos de $\mathcal{R}_n$ son los polinomios de $\mathbb{F}_q$ con grado menor que n y la multiplicación la realizamos módulo $x^n-1$. Así, cuando trabajamos en $\mathcal{R}_n$, al multiplicar dos polinomios, los multiplicamos como lo hacemos en $\mathbb{F}_q[x]$ y reemplazamos los términos de la forma $ax^{ni+j}$, con $ 0 \leq j < n$ por $ax^j$.

Para distinguir el ideal principal $(g(x))$ de $\mathbb{F}_q[x]$ del ideal principal de $\mathcal{R}_n$, denotamos $<g(x)>$ como el ideal principal de $\mathcal{R}_n$ generado por $g(x)$. Vemos ahora con el siguiente teorema que hay una correspondencia biyectiva entre los códigos cíclicos en $\mathcal{R}_n$ y los polinomios mónicos divisores de $x^n-1$.


\begin{theorem}
\label{th:base_ciclicos}
Sea $\mathcal{C} $ un código cíclico no-nulo en $\mathcal{R}_n$. Existe un polinomio $(x) \in \mathcal{C}$ que cumple las siguientes propiedades:
\begin{enumerate}
	\item $g(x)$ es el único polinomio mónico de menor grado en $\mathcal{C}$.
	\item $\mathcal{C} = <g(x)>$
	\item $g(x) | (x^n-1)$
	
	Sea $k = n - deg(g(x))$ y sea $g(x) = \sum_{i=0}^{n-k}g_ix^i$ donde $g_{n-k} = 1$. Entonces:
	\item La dimensión de $\mathcal{C}$ es k y $\{g(x),xg(x),\cdots,x^{k-1}g(x)\}$ forman una base de $\mathcal{C}$.
	\item Cada elemento de $\mathcal{C}$ se puede expresar particularmente como el producto de $g(x)f(x)$, donde $f(x) = 0$ o $deg(f(x)) < k$
	\item 	
	
	$\mathcal{G} = \begin{pmatrix}
			g_0 & g_1 & g_2 & \cdots & g_{n-k} & 0 & \cdots & 0 \\
			0 & g_0 & g_1 & g_2 & \cdots & g_{n-k} & \cdots & \vdots \\
			\vdots & \vdots & \vdots & \vdots & \vdots &\vdots & \vdots & \vdots \\
			0 & \cdots & 0 & g_0 & g_1 & g_2 & \cdots & g_{n-k} 
	\end{pmatrix} \Leftrightarrow \begin{pmatrix}
			g(x) &  &  &   \\
			 & xg(x)  &  &   \\
			 &  & \ddots &   \\
			 &  &  & x^{k-1}g(x)  \\
	\end{pmatrix}$ 
	es una matriz generadora de $\mathcal{C}$.
	\item Si $\alpha$ es la n-ésima raíz de la unidad en el cuerpo de extensión $\mathbb{F}_q^n$ entonces \[ 
	g(x) = \prod_s \mathcal{M}_{\alpha^s} (x)
	\]
	donde el producto es en un subconjunto representativo de las clases q-ciclotómicas módulo n.	
\end{enumerate}
\end{theorem}

\begin{proof}
Sea $g(x)$ un polinomio mónico de menor grado en $\mathcal{C}$. Como $\mathcal{C}$ es no-nulo, ese polinomio existe. Si $c(x) \in \mathcal{C}$, entonces por el algoritmo de la división en $\mathbb{F}_q[x]$, $c(x) = g(x)h(x) + r(x)$, donde $r(x) = $ o $deg(r(x)) < deg(g(x))$. Como $\mathcal{C}$ es un ideal en $\mathcal{R}_n$, $r(x) \in \mathcal{C}$ y como el grado de $g(x)$ es mínimo, implica que $r(x) = 0$. Esto prueba 1) y 2).

De nuevo, por el algoritmo de la división, $x^n-1 = g(x)h(x)+r(x)$, donde de nuevo $r(x) = 0$ o $deg(r(x)) < deg(g(x))$ en  $\mathbb{F}_q[x]$. Como $x^n-1$ correponde con la palbra código 0 en $\mathcal{C}$ y $\mathcal{C}$ es un ideal en $\mathcal{R}_n$, entonces $r(x) \in \mathcal{C}$ que es una contradicción, a menos que $r(x) = 0$, lo que prueba 3).

Supongamos que $deg(g(x)) = n-k$. Por 2) y 3) , si $c(x) \in \mathcal{C}$ con $c(x) = 0$ o $deg(c(x)) < n $, entonces $c(x) = g(x)f(x)$ en $\mathbb{F}_q[x]$. Si $c(x) = 0$, entonces $f(x) = 0$. Si $c(x) \neq 0$, $deg(c(x)) < n $ y el grado del producto de dos polinomio es la suma de los grados de los polinomios y sabemos que $deg(g(x)) = n-k$ lo que implica que $deg(f(x)) < k$. Por tanto, 
\[
\mathcal{C} = \{ g(x)f(x) | f(x) = 0 \thinspace o \thinspace deg(f(x)) < k \}
\]

Así que $\mathcal{C}$ tiene como mucho dimensión k y $\{g(x),xg(x),\cdots,x^{k-1}g(x)\}$ expande a $\mathcal{C}$. Como estos k polinomios son de distinto grado, son independientes en $\mathbb{F}_q[x]$. Como su grado es como mucho $n-1$, son también independientes en $\mathcal{R}_n$, por lo que queda demostrado 4) y 5). Para 6), basta colocar por filas los elementos de la base y así obtenemos $\mathcal{G}$. El último punto se obtiene del teorema \ref{factorizar_xn}. 
\end{proof}

A partir de este teorema podemos extraer el siguiente corolario.

\begin{corollary}
\label{co:generador_unico}
Sea $\mathcal{C}$ un código cíclico no-nulo en $\mathcal{R}_n$. Son equivalentes:
\begin{enumerate}
	\item $g(x)$ es el único polinomio mónico de menor grado en $\mathcal{C}$.
	\item $\mathcal{C} = <g(x)>$, $g(x)$ es mónico y $g(x) | (x^n-1)$. 
\end{enumerate}

\end{corollary}

\begin{proof}
1) implica 2) se ha demostrado en \ref{ex:base_ciclicos}. Asumiendo 2), sea $g_1(x)$ un polinomio mónico de menor grado en $\mathcal{C}$. Por la demostración del teorema \ref{ex:base_ciclicos} apartados 1) y 2), $g_1(x) | g(x)$ en $\mathbb{F}_q[x]$ y $\mathcal{C} = < g_1(x)>$. Como $ g_1(x) \in \mathcal{C} = < g(x) >$, entonces $g_1(x) = g(x)a(x) + (x^n-1)b(x)$ en $\mathbb{F}_q[x]$. Como $g(x) | (x^n-1)$, $g(x) | g(x)a(x) + (x^n-1)b(x)$ y por tanto $g(x) | g_1(x)$. Como $g_1(x)$ y $g(x)$ son mónicos  y se dividen entre ellos en $\mathbb{F}_q[x]$, luego son iguales.
\end{proof}

Del teorema, sacamos que $g(x)$ es un polinomio mónico que divide a $x^n-1$ y genera a $\mathcal{C}$. Del corolario, sacamos que además $g(x)$ es único. Luego, a este polinomio lo llamaremos el \textit{polinomio generador} del código cíclico $\mathcal{C}$.

Así que hay una correspondencia uno a uno de los códigos cíclicos no-nulos y los divisores de $x^n-1$ no iguales a $x^n-1$. Con el fin de tener una correspondencia biyectiva entre todos los códigos cíclicos de $\mathcal{R}_n$ y todos los divisores mónicos de $x^n-1$, definimos que el \textit{polinomio generador} del código cíclico cero ${0}$ sea $x^n-1$. Esto da lugar al siguiente corolario.

\begin{corollary}
El número de códigos cíclicos en $\mathcal{R}_n$ es igual a $2^m$ donde m es el número de clases q-ciclotómicas módulo n. Además, las dimensiones de los códigos cíclicos son todas las posibles sumas de los tamaños de las clases q-ciclotómicas módulo n.

\end{corollary}

\begin{exampleth}
Para el polinomio $x^9-1$ en $\mathbb{F}_2$, calculamos sus clases \textit{2-ciclotómicas} en el ejemplo \ref{ex: clase_mod_nueve} que eran $\mathcal{C}_0 = \{ 0 \}$ , $\mathcal{C}_1 = \{ 1,2,4,8,7,5 \}$ y $\mathcal{C}_3 = \{ 3,6 \}$. Luego, sus tamaños son 1,2 y 6, por tanto, por el corolario anterior sabemos que hay $2^3 = 8$ códigos cíclicos y sus dimensiones son : $0,1,2,3,6,7,8,9$ . Veamos los polinomios generadores de cada uno en la siguiente tabla.


 \begin{tabular}{ c | c | c}
	i & dimensión & $g_i(x)$ \\ \hline
	0 & 0 & $ x^9 + 1$ \\
	1 & 1 & $ (x^2+x+1)(x^6+x^3+1) = x^8+x^7+x^6+x^5+x^4+x^3+x^2+x+1$ \\ 
	2 & 2 & $(x+1)(x^6+x^3+1) = x^7+x^6+x^4+x^3+x+1$ \\
	3 & 3 & $x^6+x^3+1$ \\
	4 & 6 & $(x+1)(x^2+x+1) = x^3+1$ \\
	5 & 7 & $x^2+x+1$ \\
	6 & 8 & $x+1$ \\ 
	7 & 9 & 1

	\end{tabular}
	
\end{exampleth}


Veamos ahora un resultado con respecto a los códigos duales.

\begin{theorem}
El código dual de un código cíclico es también cíclico.
\end{theorem}

\begin{proof}
Damos primero la definición de código dual $\mathcal{C}^\perp = \left\{ \textbf{x} \in \mathbb{F}_q^n \; : \; \textbf{x} \cdot \textbf{c} = 0 \quad  \forall \textbf{c} \in \mathcal{C} \right\}$. Lo que tenemos que probar es que dado un $c \in \mathcal{C}$ entonces $xc \in \mathcal{C}^\perp$. Tomamos $c' \in \mathcal{C}^\perp$


\[
	c' \cdot y = 0 \thinspace \forall y \in \mathcal{C} \Rightarrow x \cdot c' \cdot y = 0 \Longrightarrow x \cdot c' \in \mathcal{C}^\perp
\]

Y por tanto, hemos probado que $\mathcal{C}^\perp$ es cíclico.
\end{proof}

Podemos dar la matriz generado de un código cíclico dual que, en efecto, es también la matriz de paridad de un código cíclico.

\begin{theorem}
\label{th:matriz_paridad_ciclico_dual}

Sea $\mathcal{C}$ $[n,k]$ código cíclico con polinomio generador $g(x)$. Sea $h(x)=(x^n-1)/g(x) = \sum_{i=0}^k h_ix^i$. Entonces el polinomio generador de  $\mathcal{C}^\perp$ es $g(x)^\perp = x^kh(x^{-1})/h(0)$. Además, la matriz generadora de  $\mathcal{C}^\perp$ y por tanto, la matriz de paridad de  $\mathcal{C}$ es 

$\mathcal{H} = \begin{pmatrix}
			h_k & h_{k-1} & h_{k-2} & \cdots & h_0 & 0 & \cdots & 0 \\
			0 & h_k & h_{k-1} & h_{k-2} & \cdots & h_0 & 0 & \vdots \\
			\vdots & \vdots & \vdots & \vdots & \vdots &\vdots & \vdots & \vdots \\
			0 & \cdots & h_k & h_{k-1} & h_{k-2} & \cdots & \cdots & h_0 
	\end{pmatrix}$
\end{theorem}

\begin{proof}
Como sabemos cual es el polinomio generador de $\mathcal{C}^\perp$, podemos calcular su matriz generadora.

$\begin{pmatrix}
			g(x)^\perp &  &  &   \\
			 & xg(x)^\perp  &  &   \\
			 &  & \ddots &   \\
			 &  &  & x^{k-1}g(x)^\perp  \\
	\end{pmatrix} = $
	
	$\begin{pmatrix}
			x^kh(x^{-1})/h(0) &  &  &   \\
			 & xx^kh(x^{-1})/h(0)  &  &   \\
			 &  & \ddots &   \\
			 &  &  & x^{k-1}x^kh(x^{-1})/h(0)  \\
	\end{pmatrix}$ 
	
Haciendo cuentas en $\mathcal{R}_n$ obtenemos la matriz $\mathcal{H}$. 
\end{proof}


\subsection{Idempotentes y multiplicadores}

Además del polinomio generador, podemos encontrar otros polinomios que también se pueden usar para generar un código cíclico. Otro polinomio muy común es el que llamaremos \textit{generador idempotente}.

\begin{definition}
Un elemento \textit{e} de un anillo lo llamaremos \textit{idempotente} si satisface que $e^2=e$.
\end{definition}

\begin{theorem}
Sea $\mathcal{C}$ un código cíclico en $\mathcal{R}_n$. Entonces :
\begin{enumerate}
	\item Existe un único elemento idempotente $e(x) \in \mathcal{C}$ tal que $\mathcal{C} = < e(x) >$,
	\item si $e(x)$ es un elemento idempotente no-nulo en $\mathcal{C}$, entonces $\mathcal{C} = < e(x) >$ si y solo si es una unidad de $\mathcal{C}$.
\end{enumerate}
\end{theorem} 

\begin{proof}
Sea $\mathcal{C}$ un código cero, entonces el idempotente es el polinomio cero y se verifica 1) y 2) no se puede aplicar.

Asumimos que $\mathcal{C}$ es no-nulo. Probaremos 2) primero, supongamos que $e(x)$ es una unidad en $\mathcal{C}$. Luego $<e(x)> \in \mathcal{C}$ viendo a $\mathcal{C}$ como un ideal. Si $c(x) \in \mathcal{C}$, entonces $c(x)e(x) = c(x)$ en $\mathcal{C}$, lo que implica que $<e(x)> = \mathcal{C}$. Por el contrario, supongamos que $e(x)$ es el idempotente no-nulo tal que $ \mathcal{C} = <e(x)> $. Luego, cada elemento $c(x) \in \mathcal{C}$ se puede escribir de la forma $c(x) = f(x)e(x)$, pero $c(x)e(x) = f(x)e(x)^2 = f(x)e(x) = c(x)$ lo que implica que $e(x)$ es una unidad en $\mathcal{C}$.

Como $\mathcal{C}$ es no-nulo, por 2) si $e_1(x)$ y $e_2(x)$ son generadores idempotentes, entonces ambos son unidades y $e_1(x) = e_2(x)e_1(x) = e_2(x)$. Solo nos falta probar la existencia. Si $g(x)$ es el polinomio generador de $\mathcal{C}$, entonces $g(x) | x^n-1$. Sea $h(x) = (x^n-1)/g(x)$, entonces el $mcd(g(x),h(x)) = 1$ en $\mathbb{F}_q[x]$ ya que $x^n-1$ tiene distintas raíces. Por el Algoritmo de Euclídes, sabemos que existen $a(x),b(x) \in \mathbb{F}_q[x]$ tal que $a(x)g(x) + b(x)h(x) = 1$. Sea $e(x) \equiv a(x)g(x) \thinspace (mod \thinspace x^n-1)$, donde $e(x)$ es el representante de $a(x)g(x) + (x^n-1)$ en $\mathcal{R}_n$. Luego en $\mathcal{R}_n$, $e(x)^2 \equiv (a(x)g(x))(1-b(x)h(x)) \equiv a(x)g(x) \equiv e(x) \thinspace (mod \thinspace x^n-1)$ ya que $g(x)h(x) = x^n-1$. Además si $c(x) \in \mathcal{C}$, $c(x) = f(x)g(x)$ implica que $c(x)e(x) \equiv f(x)g(x)(1-b(x)h(x)) \equiv f(x)g(x) \equiv c(x) \thinspace (mod \thinspace x^n-1)$, por tanto, $e(x)$ es una unidad de $\mathcal{C}$ y 1) se prueba a partir de 2).
\end{proof}

Gracias a la demostración, hemos encontrado una forma de calcular el polinomio $e(x)$ a partir del polinomio generador $g(x)$. Basta con resolver $1 = a(x)g(x) + b(x)h(x)$ donde $h(x) = (x^n-1)/g(x)$. Luego, reduciendo $a(x)g(x)$ módulo $x^n-1$ se tiene $e(x)$. Veamos ahora una forma de obtener $g(x)$ a partir de $e(x)$.

\begin{theorem}
Sea $\mathcal{C}$ un código cíclico sobre $\mathbb{F}_q$ con generador idempotente $e(x)$. Entonces, el polinomio generador de $\mathcal{C}$ es $g(x) = mcd(e(x),x^n-1)$ en $\mathbb{F}_q[x]$.
\end{theorem}

\begin{proof}
Sea $d(x) = mcd(e(x),x^n-1)$ en $\mathbb{F}_q[x]$ y sea $g(x)$ el polinomio generador de $\mathcal{C}$. Como $d(x) | e(x)$, $e(x) = d(x)k(x)$ implica que cada elemento de $\mathcal{C} = < e(x) > $ es también un múltiplo de $d(x)$, así que $\mathcal{C} \subseteq <d(x)<$. Por el teorema \ref{ex:base_ciclicos}, sabemos que $g(x) | (x^n-1)$ y por tanto, $g(x) | e(x)$ porque $e(x) \in \mathcal{C}$. Luego $g(x) | d(x)$ y por tanto, $d(x) \in \mathcal{C}$. Como $d(x)$ es un divisor mónico de $x^n-1$ que genera a $\mathcal{C}$, entonces por el corolario \ref{co:generador_unico}, $d(x) = g(x)$.
\end{proof}

\begin{exampleth}
\label{ex:idempotente_siete}
Vamos a calcular las clases \textit{2-ciclotómicas} para $n=7$ y ver cuáles son sus códigos cíclicos, dando su polinomio generador e idempotente.

Las clases \textit{2-ciclotómicas} son : $\mathcal{C}_0 = \{ 0 \} $, $\mathcal{C}_1 = \{ 1*2^r \equiv 1 \thinspace (mod \thinspace 7) \} = \{ 1,2,4 \} $ , $\mathcal{C}_3 = \{ 3*2^r \equiv 3 \thinspace (mod \thinspace 7) \} = \{ 3,6,5 \} $. 

Por tanto, tenemos tres clases \textit{2-ciclotómicas} de tamaños, 1,3 y 3 y la factorización de $x^7-1 = (x+1)(x^3+x^2+1)(x^3+x+1)$ Sabemos que hay 8 códigos cíclicos cuyas dimensiones son 0,1,3,3,4,4,6,7. En la siguiente tabla veremos los polinomios generadores e idempotentes.

 \begin{tabular}{ c | c | c | }
	i & dimensión & $g_i(x)$ \\ \hline
	0 & 0 & $ x^7 + 1$  \\
	1 & 1 & $ (x^3+x+1)(x^3+x^3+1) = x^6+x^5+x^4+x^3+x^2+x+1$  \\ 
	2 & 3 & $ (x+1)(x^3+x^2+1) = x^4+x^2+x+1$ \\
	3 & 3 & $ (x+1)(x^3+x+1) = x^4+x^3+x^2+1$  \\
	4 & 4 & $ x^3+x^2+1$  \\
	5 & 4 & $ x^3+x+1$ \\
	6 & 6 & $x+1$  \\ 
	7 & 7 & 1 

	\end{tabular}
	
	\begin{tabular}{ c | c | c |}
	i & dimensión & $e_i(x)$\\ \hline
	0 & 0  & 0 \\
	1 & 1 & $x^6+x^5+x^4+x^3+x^2+x+1$ \\ 
	2 & 3  & $x^6+x^5+x^3+1$\\
	3 & 3  &  $x^4+x^3+x^2+1$ \\
	4 & 4  & $x^4+x^2+x$ \\
	5 & 4  & $x^6+x^5+x^3$\\
	6 & 6  & $x^6+x^5+x^4+x^3+x^2+x$ \\ 
	7 & 7  & 1

	\end{tabular}

Los polinomios idempotentes los hemos calculado como indicamos anteriormente. Además, los dos códigos de dimensión cuatro son los $[7,4,3]$ códigos Hamming.

\end{exampleth}


Veamos ahora que al igual que los polinomios generadores podíamos sacar la matriz generadora, también podemos hacerlo con los idempotentes.

\begin{theorem}
Sea $\mathcal{C}$ un $[n,k]$ código cíclico con polinomio idempotente $e(x)= \sum_{i=0}^{n-1} e_ix^i$, la matriz $k \times n $

$ \begin{pmatrix}
			e_0 & e_1 & e_2 & \cdots & e_{n-2} & e_{n-1} \\
			e_{n-1} & e_0 & e_1 & \cdots  & e_{n-3} & e_{n-2} \\
			\vdots & \vdots & \vdots & \vdots & \vdots &\vdots & \vdots & \vdots \\
			e_{n-k+1} & e_{n-k+2} & e_{n-k+3} & \cdots  & e_{n-k-1} & e_{n-k} \\
	\end{pmatrix}$

es la matriz generadora de $\mathcal{C}$.
\end{theorem}

\begin{proof}
Esto es equivalente a decir que $\{ e(x),xe(x),\cdots,x^{k-1}e(x) \}$ es una base de $\mathcal{C}$. Por tanto,  es suficiente ver que si $a(x) \in \mathbb{F}_q[x]$ tiene menor grado que \textit{k} tal que $a(x)e(x)=0$, entonces $a(x)=0$. Sea $g(x)$ el polinomio generador de $\mathcal{C}$. Si $a(x)e(x)=0$, entonces $0 = a(x)e(x)g(x)= a(x)g(x)$ ya que $e(x)$ es una unidad de $\mathcal{C}$, contradiciendo así el teorema \ref{th:base_ciclicos} 5) a menos que $a(x) = 0$.
\end{proof}


\begin{definition}
Sea $\mathcal{C}_1$ y $\mathcal{C}_2$ códigos de tamaño \textit{n} en $\mathbb{F}_q$, definimos la suma de $\mathcal{C}_1$ y $\mathcal{C}_2$ como 
\[
\mathcal{C}_1 + \mathcal{C}_2 = \{ c_1 + c_2 \mid \thinspace c_1 \in \mathcal{C}_1 \thinspace y \thinspace c_2 \in \mathcal{C}_2 \}
\]
\end{definition}

\begin{theorem}
\label{th:inter_suma_generadores}
Sea $\mathcal{C}_i$ un código cíclico de longitud n en $\mathbb{F}_q$ con polinomio generador $g_i(x)$ y generador idempotente $e_i(x)$ con $i = 1,2$. Entonces:
\begin{enumerate}
	\item $\mathcal{C}_1 \cap \mathcal{C}_2$ tiene como polinomio generador el $mcm(g_1(x),g_2(x))$ y generador idempotente $e_1(x)e_2(x)$
	\item $\mathcal{C}_1 + \mathcal{C}_2$ tiene como polinomio generador el $mcd(g_1(x),g_2(x))$ y generador idempotente $e_1(x) + e_2(x) - e_1(x)e_2(x)$
\end{enumerate}
\end{theorem}

%TODO demostración viene hecha la de la suma y la de la intersección me falta una inclusión, página 135 del libro 

Describiremos ahora un tipo especial de idempotentes que son los \textit{idempotentes primitivos}, los cuales, una vez conocidos, podemos obtener todos los idempotentes de $\mathcal{R}_n$ y por tanto, todos los códigos cíclicos.

\begin{definition}
Sea $x^n-1 = f_1(x) \cdots f_s(x)$, donde $f_i(x)$ es irreducible en $\mathbb{F}_q$ para $ 1 \leq i \leq s$. Sea $\widehat{f_i}(x) = (x^n-1)/f_i(x)$, llamaremos \textit{idempotentes primitivos} a $\widehat{e_i}(x)=<\widehat{f_i}(x)>$
\end{definition}

\begin{theorem}
\label{th:idempotentes primitivos}
Lo siguiente es cierto en $\mathcal{R}_n$ :
\begin{enumerate}
	\item Los ideales $<\widehat{f_i}(x)>$ para $ 1 \leq i \leq s$ son todos minimales en $\mathcal{R}_n$
	\item $\mathcal{R}_n$ es la suma directa de $<\widehat{f_i}(x)>$ para $ 1 \leq i \leq s$
	\item Si $ i \neq j $, entonces $\widehat{e_i}(x)\widehat{e_j}(x) = 0$ en $\mathcal{R}_n$
	\item $\sum_{i=1}^s \widehat{e_i}(x) = 1$ en $\mathcal{R}_n$
	\item Los únicos elementos idempotentes en $<\widehat{f_i}(x)>$ son 0 y $\widehat{e_i}(x)$
	\item Si $e(x)$ es un idempotente no-nulo en $\mathcal{R}_n$, entonces hay un subconjunto $T$ de $\{ 1,2, \cdots , s \}$ tal que $e(x) = \sum_{i \in T} \widehat{e_i}(x) $ y $ < e(x) > = \sum_{i \in T} < \widehat{f_i}(x) > $.
\end{enumerate}
\end{theorem}

\begin{proof}
Supongamos que los $<\widehat{f_i}(x)>$ no son ideales minimales en $\mathcal{R}_n$. Luego, tiene que haber un polinomio generador $g(x)$ de un ideal no-nulo contenido en $<\widehat{f_i}(x)>$ tal que $f_i(x) | g(x) $ con $g(x) \neq f_i(x)$. Como $f_i(x)$ es irreducible y $ g(x)| (x^n-1) $, esto es imposible. Así que $<\widehat{f_i}(x)>$ es el ideal minimal de $\mathcal{R}_n$, con esto ya tenemos parte de 1).

Como $\{ \widehat{f_i}(x) | 1 \leq i \leq s \}$ no tiene factores irreducibles comunes de $x^n-1$ y cada polinomio en el conjunto divide a $x^n-1$, $mcd(\widehat{f_1}(x), \cdots , \widehat{f_s}(x) ) = 1 $. Aplicando el Algoritmo de Euclides, tenemos que 
\begin{equation}
	1 = \sum_{i=1}^s a_i(x)\widehat{f_i}(x)
\end{equation}

para algunos $a_i(x) \in \mathbb{F}_q[x]$. Así que 1 es la suma de todos los ideales $<\widehat{f_i}(x)>$  que es también un ideal de $\mathcal{R}_n$. En cualquier anillo, el único ideal que contiene a la identidad del anillo es el propio anillo, luego esto prueba que $\mathcal{R}_n$ es la suma de los ideales $<\widehat{f_i}(x)>$. Para ver que es directa, debemos probar que $<\widehat{f_i}(x)> \cap \sum_{j \neq i} <\widehat{f_j}(x)> = { 0 }$ para $1 \leq i \leq s$. Como $f_i(x) | \widehat{f_j} $ para $j \neq i $, $f_i(x) \nmid \widehat{f_j} $ y los factores irreducibles de $x^n-1$ on distintos, luego $f_i(x) = mcd \{ \widehat{f_j}(x) | 1 \leq i \leq s \thinspace , j \neq i \}$ . Aplicando inducción llegamos a que $<\widehat{f_i}(x)> = \sum_{j \neq i} <\widehat{f_j}(x)> $. Así que $<\widehat{f_i}(x)> \cap \sum_{j \neq i} <\widehat{f_j}(x)> = <\widehat{f_i}(x)> \cap <f_i(x)> = <mcm(\widehat{f_i}(x),f_i(x))> = <x^n-1> = { 0 }$ probando así 2).

Sea $\mathcal{M} = <m(x)>$ un ideal minimal cualquiera de  $\mathcal{R}_n$. Tenemos que 
\[
0 \neq m(x) = m(x) \cdot 1 = \sum_{i=1}^s m(x)a_i(x)\widehat{f_i}(x)
\]

luego, hay un \textit{i} tal que $m(x)a_i(x)\widehat{f_i}(x) \neq 0$. Por tanto, $\mathcal{M} \cap <\widehat{f_i}(x)> \neq {0}$ ya que $m(x)a_i(x)\widehat{f_i}(x) \in \mathcal{M} \cap <\widehat{f_i}(x)>$ y además $\mathcal{M} = <\widehat{f_i}(x)>$ por ser $\mathcal{M}$ y $<\widehat{f_i}(x)>$ minimales. Esto completa la prueba de 1).

Si $ i \neq j$, $\widehat{e_i}(x)\widehat{e_j}(x) \in <\widehat{f_i}(x)> \cap <\widehat{f_j}(x)> = {0}$ por 2), lo que prueba 3). Usando 3) e inducción en el apartado 2) del teorema \ref{th:inter_suma_generadores} , $\sum_{i=1}^s \widehat{e_i}(x)$ es el generador idempotente de $\sum_{i=1}^s \widehat{f_i}(x) = \mathcal{R}_n$ por 2) y el generador idempotente de $\mathcal{R}_n$ es 1, luego hemos probado 4). 

Si $e(x)$ es un idempotente no-nulo en $<\widehat{f_i}(x)>$, entonces $<e(x)>$ es un ideal contenido en $<\widehat{f_i}(x)>$. Por ser minimal y $e(x)$ no nulo, tenemos que $<\widehat{f_i}(x)> = <e(x)>$, implicando que $e(x) = \widehat{e_i}(x)$ ya que ambos son las únicas unidades de $\widehat{f_i}(x)$. Así tenemos 5).

Para 6), notamos que $e(x)\widehat{e_i}(x)$ es idempotente en $\widehat{f_i}(x)$, luego $e(x)\widehat{e_i}(x)$ es 0 o es $\widehat{e_i}(x)$ por 5). Sea $T = \{ i | \thinspace e(x)\widehat{e_i}(x) \neq 0 \}$. Entonces, por 4), $e(x) = e(x) \cdot 1 = e(x)\sum_{i=1}^s \widehat{e_i}(x) = \sum_{i=1}^s e(x)\widehat{e_i}(x) = \sum_{i \in T} \widehat{e_i}(x)$. Además, $<e(x)> = < \sum_{i \in \mathcal{T}} \widehat{e_i}(x)> = \sum_{i \in T} <\widehat{e_i}(x)>$ por el teorema \ref{th:inter_suma_generadores} 2) e inducción, probándose así 6).
\end{proof}

\begin{theorem}
Sea $\mathcal{M}$ un ideal minimal de $\mathcal{R}_n$. Entonces $\mathcal{M}$ es un cuerpo de extensión de $\mathbb{F}_q$.
\end{theorem}

\begin{proof}
Solo hay que probar que cada elemento no-nulo en $\mathcal{M}$ tiene un inverso multiplicativo en $\mathcal{M}$. Sea $a(x) \in \mathcal{M}$ no-nulo. Entonces $<a(x)>$ es un ideal no-nulo de $\mathcal{R}_n$ contenido en $\mathcal{M}$, y por tanto, $<a(x)> = \mathcal{M}$. Así que, si $e(x)$ es la unidad en $\mathcal{M}$, hay un elemento $b(x)$ en $\mathcal{R}_n$ tal que $a(x)b(x) = e(x)$. Ahora si $c(x) = b(x)e(x) \in \mathcal{M}$ con $e(x) \in \mathcal{M}$. Por tanto, $a(x)c(x) = e(x)^2 = e(x) $
\end{proof}

Veremos ahora una permutación particular que nos permite mapear idempotentes de $\mathcal{R}_n$ en idempotentes de $\mathcal{R}_n$.

\begin{definition}
Sea \textit{a} un entero tal que $mcd(a,n) = 1$. La función $\mu_a$ definida en $ \{ 0,1 , \cdots , n-1 \}$ por $i\mu_a \equiv ia \thinspace (mod \thinspace n)$ es una permutación de las coordenadas $ \{ 0,1 , \cdots , n-1 \}$ de un código cíclico de tamaño n y la llamaremos \textit{multiplicador}.
\end{definition}

Otra forma de ver a los multiplicadores es con la ecuación 
\begin{equation}
f(x)\mu_a \equiv f(x^a) \thinspace (mod \thinspace x^n-1)
\end{equation}

es consistente con la otra definición ya que $x^i\mu_a = x^{ia} = x^{ia + jn}$ en $\mathcal{R}_n$ para un entero \textit{j} tal que $ 0 \leq ia + jn < n$ ya que $x^n = 1 $ en $\mathcal{R}_n$. En otras palabras, $x^i\mu_a = x^{ia \thinspace mod \thinspace n}$.


\begin{theorem}
\label{th:propiedades de multiplicadores}
Sean $f(x)$ y $g(x)$ elementos de $\mathcal{R}_n$. Supongamos que $e(x)$ es idempotente en $\mathcal{R}_n$ y sea a primo relativo con n. Entonces : 
\begin{enumerate}
	\item Si $ b \equiv a \thinspace (mod \thinspace n)$ entonces $\mu_b = \mu_a$
	\item $(f(x) + g(x))\mu_a = f(x)\mu_a + g(x)\mu_a$
	\item $(f(x)g(x))\mu_a = (f(x)\mu_a)(g(x)\mu_a)$
	\item $\mu_a$ es un automorfismo de $\mathcal{R}_n$
	\item $e(x)\mu_a$ es idempotente en $\mathcal{R}_n$
	\item $\mu_q$ deja invariante cada clase q-ciclotómica módulo n y tiene orden igual al $ord_n(q)$
\end{enumerate}
\end{theorem}


\begin{theorem}
Sea $\mathcal{C}$ un código cíclico de longitud n sobre $\mathbb{F}_q$ con generador idempotente $e(x)$. Sea a un entero con $mcd(a,n) = 1$. Entonces :
\begin{enumerate}
	\item $\mathcal{C}\mu_a = <e(x)\mu_a>$ y $e(x)\mu_a$ es el generador idempotente de $\mathcal{C}\mu_a$
	\item $ e(x)\mu_q = e(x)$ y $\mu_q \in PAut(\mathcal{C})$
\end{enumerate}
\end{theorem}


\begin{proof}
Usando el teorema \ref{th:propiedades de multiplicadores} 3), $\mathcal{C}\mu_a = \{ (e(x)f(x))\mu_a \thinspace | \thinspace f(x) \in \mathcal{R}_n \} = \{ e(x)\mu_a \times f(x)\mu_a \thinspace | \thinspace f(x)\mu_a \in \mathcal{R}_n \} = \{ e(x)\mu_ah(x) \thinspace | \thinspace h(x) \in \mathcal{R}_n \} = <e(x)\mu_a> $ ya que $\mu_a$ es un automofirsmo por el teorema \ref{th:propiedades de multiplicadores} 4). Por tanto, $\mathcal{C}\mu_a$ es cíclico y tiene de generador idempotente $e(x)\mu_a$ por el teorema \ref{th:propiedades de multiplicadores} 5), probando así 1).

Si probamos que $e(x)\mu_q = e(x)$, luego por 1), $\mathcal{C}\mu_q = \mathcal{C}$ y por tanto $\mu_q \in PAut(\mathcal{C})$. Por el teorema \ref{th:idempotentes primitivos} 6), $e(x) = \sum_{i \in T} \widehat{e_i}(x) $ para algún conjunto T. Por el teorema \ref{th:propiedades de multiplicadores} 2), $e(x)\mu_q = e(x)$ si $\widehat{e_i}(x)\mu_q = \widehat{e_i}(x)$ para todo \textit{i}. Pero $\widehat{e_i}(x)\mu_q = \widehat{e_i}(x^q) = (\widehat{e_i}(x))^q$ siendo un elemento no-nulo de $<\widehat{e_i}(x)>$ luego por el teorema \ref{th:idempotentes primitivos} 5) $\widehat{e_i}(x)\mu_q  = \widehat{e_i}(x)$.
\end{proof}

\subsection{Ceros de un código cíclico}

\begin{definition}
Sea $\mathcal{C}$ un código cíclico en $\mathcal{R}_n$ con polinomio generador $g(x)$. Luego, $g(x) = \prod_{s} \mathcal{M}_{\alpha^s}(x) = \prod_{s} \prod_{i \in \mathcal{C}_s} (x-\alpha^i) $ donde \textit{s} recorre un subconjunto de los representantes de las clases \textit{q-ciclotómicas} $\mathcal{C}_s$ módulo n. Sea T $ = \cup_s \mathcal{C}_s$ la unión de estas clases \textit{q-ciclotómicas}. Llamaremos \textbf{ceros} de un código cíclico a $\mathcal{Z} = \{ \alpha^i \thinspace | \thinspace i \in T \}$ y \textbf{no-ceros} a $\{ \alpha^i \thinspace | \thinspace i \not\in T \}$ . Al conjunto T, lo llamaremos \textbf{conjunto de definición} de $\mathcal{C}$
\end{definition}

Nos damos cuenta de que T y por tanto, el conjunto de ceros y no-ceros, determina completamente al polinomio generador $g(x)$. Por el teorema \ref{th:base_ciclicos} , la dimensión de $\mathcal{C}$ es $ n - \mid T \mid$ y $\mid T \mid$ es el grado de $g(x)$


\begin{exampleth}
Utilizaremos los polinomios generadores e idempotentes calculados en el ejemplo \ref{ex:idempotente_siete} pero además, mostraremos el conjunto de definición para cada código cíclico relativo a la raíz primitiva $\alpha$ 

\begin{tabular}{ c | c | c | }
	i & dimensión & $g_i(x)$ \\ \hline
	0 & 0 & $ x^7 + 1$  \\
	1 & 1 & $  x^6+x^5+x^4+x^3+x^2+x+1$  \\ 
	2 & 3 & $  x^4+x^2+x+1$ \\
	3 & 3 & $  x^4+x^3+x^2+1$  \\
	4 & 4 & $ x^3+x^2+1$  \\
	5 & 4 & $ x^3+x+1$ \\
	6 & 6 & $x+1$  \\ 
	7 & 7 & 1 

	\end{tabular}
	
	\begin{tabular}{ c | c | c | c |}
	i & dimensión & $e_i(x)$ & conjunto de definición\\ \hline
	0 & 0  & 0 & $\{ 0,1,2,3,4,5,6 \}$ \\
	1 & 1 & $x^6+x^5+x^4+x^3+x^2+x+1$  & $\{ 1,2,3,4,5,6 \}$\\ 
	2 & 3  & $x^6+x^5+x^3+1$ & $\{ 0,1,2,4 \}$\\
	3 & 3  &  $x^4+x^3+x^2+1$ & $\{ 0,3,5,6 \}$ \\
	4 & 4  & $x^4+x^2+x$ & $\{ 1,2,4 \}$ \\
	5 & 4  & $x^6+x^5+x^3$ & $\{ 3,5,6 \}$ \\
	6 & 6  & $x^6+x^5+x^4+x^3+x^2+x$ & $\{ 0 \}$ \\ 
	7 & 7  & 1 & $ \emptyset $

	\end{tabular}
\end{exampleth}

El siguiente teorema nos da propiedades básicas de los códigos cíclicos en relación a los conjuntos de definición.

\begin{theorem}
\label{th:prop_conj_def}

Sea $\alpha$ la n-ésima raíz primitiva de la unidad en un cuerpo de extensión de $\mathbb{F}_q$. Sea $\mathcal{C}$ un código cíclico de longitud n en $\mathbb{F}_q$ con conjunto de definición T y polinomio generador $g(x)$. Se verifica lo siguiente:
\begin{enumerate}
	\item T es la unión de las clases q-ciclotómicas módulo n
	\item $g(x) = \prod_{i \in T} (x-\alpha^i)$
	\item $c(x) \in \mathcal{R}_n \Leftrightarrow \thinspace c(\alpha^i) = 0 \thinspace \forall i \in T$
 	\item La dimensión de $\mathcal{C}$ es $ n - \mid T \mid$
\end{enumerate}
\end{theorem}


Los ceros de un código cíclico nos pueden servir para obtener una matriz de control de paridad, como la definimos en el siguiente teorema.

\begin{theorem}
Sea $\mathcal{C}$ un $[n,k]$ código cíclico sonre $\mathbb{F}_q$ con ceros $\mathcal{Z}$ en el cuerpo de extensión $\mathbb{F}_q^t$ de $x^n-1$ sobre $\mathbb{F}_q$. Sea $ \alpha \in \mathbb{F}_q^t$ la n-ésima raíz primitiva de la unidad en $\mathbb{F}_q^t$ y sea $\mathcal{Z} = \{ \alpha^j \thinspace | \thinspace j \in \mathcal{C}_{i_1} \cup \cdots \cup \mathcal{C}_{i_w} \} $, donde $\mathcal{C}_{i_1}, \cdots , \mathcal{C}_{i_w}$ son las distintas clases q-ciclotómicas módulo n. Sea L la matriz $w \times n$ sobre $\mathbb{F}_q^t$ definida por:

$ L = \begin{pmatrix}
			1 & \alpha^{i_1} & \alpha^{2i_1} & \cdots & \alpha^{(n-1)i_1} \\
			1 & \alpha^{i_2} & \alpha^{2i_2} & \cdots  & \alpha^{(n-1)i_2} \\
			\vdots & \vdots & \vdots & \vdots & \vdots  \\
			1 & \alpha^{i_w} & \alpha^{2i_w} & \cdots  & \alpha^{(n-1)i_w} \\
	\end{pmatrix}$
	
	
Entonces, \textbf{c} está en $\mathcal{C} \Leftrightarrow Lc^T = 0$. Escogiendo una base de $\mathbb{F}_q^t$ sobre $\mathbb{F}_q$, podemos representar cada elemento de $\mathbb{F}_q^t$ como el vector columna $t \times 1$ sobre $\mathbb{F}_q$. Reemplazando cada entrada de L por su correspondiente vector columna, obtenemos una $tw \times n$  matriz H sobre $\mathbb{F}_q$, que tiene la propiedad de que  $ c \in \mathcal{C} \Leftrightarrow Hc^T = 0$. En particular, $ k \geq n - tw$.
\end{theorem}

\begin{proof}
Tenemos que $ c(x) \in \mathcal{C} \Leftrightarrow c(\alpha^j) = 0 \thinspace \forall j \in \mathcal{C}_{i_1} \cup \cdots \cup \mathcal{C}_{i_w}$ y esto es equivalente a ver que $c(\alpha^{i_j})$ para $1 \leq j \leq w$. Además, esto es equivalente a ver que $Lc^T = 0$, que es un sistema de ecuaciones lineales homogéneo cuyos coeficientes son las potencias de $\alpha$. Expandiendo cada una de estas potencias en la base escogida de $\mathbb{F}_q^t$ sobre $\mathbb{F}_q$ nos da un sistema equivalente $Hc^T = 0$. Como las filas de H puede que sean dependientes, $k \geq n -tw$.
\end{proof}

Para cada código cíclico $\mathcal{C}$ en $\mathcal{R}_n$, hay varios polinomios $v(x)$ tal que $\mathcal{C} = < v(x) >$, pero hemos visto que solo hay uno que sea mónico y de mínimo grado al que llamamos polinomio generador. En el siguiente teorema vamos a caracterizar todos los polinomios $v(x)$ que generan a $\mathcal{C}$.

\begin{theorem}
Sea $\mathcal{C}$ un código cíclico de longitud n sobre $\mathbb{F}_q$ con polinomio generador $g(x)$. Sea $v(x)$ un polinomio en $\mathcal{R}_n$.
\begin{enumerate}
	\item $\mathcal{C} = <v(x)> \Leftrightarrow \thinspace mcd(v(x),x^n-1) = g(x)$
	\item $v(x)$ genera $\mathcal{C} \Leftrightarrow $ las n-ésimas raíces de la unidad que son ceros de $v(x)$ son precisamente los ceros de $\mathcal{C}$
\end{enumerate}
\end{theorem}


\begin{proof}
Primero asumimos que $mcd(v(x),x^n-1) = g(x)$. Como $g(x) | v(x)$, los múltiplos de $v(x)$ son múltiplos de $g(x)$ en $\mathcal{R}_n$ y por tanto $<v(x)> \subseteq \mathcal{C}$. Por el algoritmo de Euclides, existen los polinomios $a(x)$ y $b(x)$ en $\mathbb{F}_q[x]$ tal que $g(x) = a(x)v(x) + b(x)(x^n-1)$. Por tanto, $g(x) = a(x)v(x)$ está en  $\mathcal{R}_n$ y además, los múltiplos de $g(x)$ son múltiplos de $v(x)$ en $\mathcal{R}_n$ lo que implica que $\mathcal{C} \subseteq <v(x)>$ . Por tanto, $\mathcal{C} = <v(x)>$.

Por el contrario, ahora asumimos que $\mathcal{C} = <v(x)>$. Sea $d(x) =  mcd(v(x),x^n-1)$. Como $g(x) | v(x)$ y $g(x) | (x^n-1) $, enotnces $g(x) | d(x)$. Como $g(x) \in \mathcal{C} = <v(x)>$, existe un polinomio $a(x)$ tal que $g(x) = a(x)v(x)$ está en  $\mathcal{R}_n$. Así que existe un polinomio $b(x)$ tal que  $g(x) = a(x)v(x) + b(x)(x^n-1)$ en $\mathbb{F}_q[x]$. Así, $d(x) | g(x)$. Tanto como $d(x)$ como $g(x)$ son mónicos y se dividen entre ellos, $d(x) = g(x)$ y hemos probado 1).

Como las únicas raíces tanto de $g(x)$ como de $x^n-1$ son las n-ésimas raíces  de la unidad, $g(x) = mcd(v(x),x^n-1) \Leftrightarrow  $ las n-ésimas raíces de la unidad que son ceros de $v(x)$ son precisamente los ceros de $g(x)$ que son los mismo que los ceros de $\mathcal{C}$ y así hemos probado 2).

\end{proof}

\begin{corollary}
Sea $\mathcal{C}$ un código cíclido de longitud n sobre $\mathbb{F}_q$ con ceros $\{ \alpha^i \thinspace | \thinspace i \in T \}$ para alguna n-ésima raíz primitiva $\alpha$ donde T es un conjunto de definición de $\mathcal{C}$. Sea a un entero tal que $mcd(a,n)=1$ y sea $a^{-1}$ el inverso multiplicativo de a en los enteros módulo n. Entonces  $\{ \alpha^{a^{-1}i} \thinspace | \thinspace i \in T \}$ son los ceros del código cíclico  $\mathcal{C}\mu_a$ y $a^{-1}T \thinspace mod \thinspace n$ es le conjunto de definición para $\mathcal{C}\mu_a$,
\end{corollary}

 \begin{proof}
 Sea $e(x)$ el generador idempotente $\mathcal{C}$ . Por el teorema \ref{th:idempotentes primitivos}, el generador idempotente de un código cíclico $\mathcal{C}\mu_a$ es $e(x)\mu_a$. Por el teorema anterior, los ceros de $\mathcal{C}$  y de $\mathcal{C}\mu_a$ son las n-ésimas raíces de la unidad que también son raíces de $e(x)$ y $e'(x) = e(x)\mu_a$, respectivamente. Como  $e'(x) = e(x)\mu_a \equiv e(x^a) \thinspace (mod \thinspace x^n-1), e'(x) = e(x^a) + b(x)(x^n-1)$ en $\mathbb{F}_q[x]$. Terminamos sabiendo que la n-ésima raíz de la unidad $\alpha^j$ es una raíz de $e'(x) \Leftrightarrow \alpha^{aj}$ es raíz de $e(x)$.
 \end{proof}
 
 
 
 
 
 \subsection{Mínima distancia de códigos cíclicos}
 
 Es importante saber cual es la mínima distancia que puede alcanzar cualquier código, para así poder determinar su capacidad de corrección de errores. Luego, es útil encontrar cotas inferiores para la mínima distancia. Una de las cotas más antiguas es la cota de \textit{Bose-Ray-Chaudhuri-Hocquenghem}, normalmente conocida como la \textit{cota BCH} que nos servirá para definir los códigos BCH más adelante. 
 
 Antes de definir la cota BCH, vamos a dar un lema sobre el determinante de la matriz de Vandermonde que nos servirá para probar dicha cota.
 
 \begin{lemma}
 \label{le:vandermonde}
 $det \thinspace V = \prod_{1 \leq i < j \leq s} (\alpha_j - \alpha_i)$ . En particular, V es regular si los elementos $\alpha_1, \cdots, \alpha_s$ son distintos.
 \end{lemma}


