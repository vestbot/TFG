
% Capitulo final
% Conclusiones del trabajo


\chapter{Conclusiones y vías futuras}

La teoría de códigos es una vía de investigación importante. 
El objetivo final de este trabajo que era el estudio de la teoría de códigos lineales, el anillo de los polinomios torcidos y la unión de ambos resultados para el estudio de los códigos torcidos se ha alcanzado satisfactoriamente, es decir, se han adquirido los conocimientos necesarios para lograr estudiar e implementar la construcción de estos códigos para el cual se ha hecho un desarrollo previo de conceptos como son los cuerpos finitos, los códigos lineales y el anillo de los polinomios torcidos. También se ha llevado a cabo el desarrollo de un algoritmo de detección y correción de códigos y que su implementación en SageMath funcione correctamente.

Como vías futuras para ampliar este trabajo tenemos muchas opciones, ya que existen más tipo de códigos correctores como pueden ser los códigos de Goppa o los códigos de Hamming. Además, existen más tipos de cotas para calcular la distancia mínima de un código por lo que también sería interesante estudiarlas. En \cite{Huffman_Pless_2010} se exponen otros tipos de códigos así como más cotas y otros algoritmos de decodificación diferentes al de Sugiyama.

Otras posibilidades las podemos encontrar en investigar para los códigos torcidos ya que esta área no está muy desarrollada. Podríamos utilizar algunos de los códigos mencionados anteriormente y ver cómo se construyen en el anillo de los polinomios torcidos.

%Además, sería interesante contribuir con las implementaciones de códigos torcidos o Reed-Solomon torcidos en Sagemath ya que no dispone nada de esto.



