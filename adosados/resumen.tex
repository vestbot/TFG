


\chapter*{Resumen}

La teoría de códigos es una disciplina que se ha ido desarrollando cada vez más en los últimos años tanto en el ámbito matemático como en el informático y que tiene sobre todo gran importancia en el área de la comunicación. Su objetivo es poder corregir los posibles errores que ocurren durante la transmisión de un mensaje por un canal con ruido. Por ejemplo, si hemos enviado un robot al planeta Marte y queremos recibir imágenes antes de que se agote su batería o pilas, es algo delicado ya que si por el camino estas imágenes se ven influenciadas por ondas electromagnéticas podrían llegar distorsionadas y no servirían de nada lo que haría que la misión fracasase.

Con el fin de detectar y corregir errores se han desarrollado códigos junto a algoritmos de codificación y decodificación de estos. En este trabajo se ha estudiado la teoría básica de códigos lineales, en concreto, los códigos cíclicos ya que los elementos de este tipo de códigos los podemos expresar de forma polinómica lo que hace que trabajar con ellos sea más sencillo. Además, se estudiará una familia de códigos importante como son los códigos BCH y los códigos Reed-Solomon ya que por su construcción, se obtendrán códigos con distancia mínima elevada lo que resultará en una mejor capacidad de corrección de errores. También se verá el Algoritmo de Sugiyama que tras recibir un mensaje con errores seremos capaces de obtener el original. 

Toda esta teoría se desarrollará también para polinomios torcidos, se estudiará el anillo de los polinomios torcidos con su definición y propiedades ya que trabajamos en un ambiente no conmutativo por tanto, tenemos divisiones a ambos lados, factorización no única y una nueva forma de evaluar los polinomios. Así, con este anillo definiremos el concepto de código torcido y se estudiará una familia de códigos que son los códigos Reed-Solomon torcidos que también tendrán un peso mínimo alto. Por último, se estudiará el Algoritmo de Sugiyama para estos tipos de códigos que tiene un ligero cambio al que se ha estudiado en secciones anteriores ya que en ciertos casos no es capaz de calcular el error, por tanto, se estudiará también qué hacer en caso de que esto ocurra.

Finalmente, se implementará este algoritmo para códigos Reed-Solomon torcidos además del caso en el que no podemos encontrar el error para comprobar e ilustrar que esta teoría funciona correctamente y la potencia que tiene. 

\textbf{Palabras clave:} teoría de códigos,códigos cíclicos, códigos torcidos, Reed-Solomon, Algoritmo de Sugiyama, decodificación, polinomios torcidos, correción errores. 
